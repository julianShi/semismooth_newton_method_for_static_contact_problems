%%%%%%%%%%%%%%%%%%%%%%%%%%%%%%%%%%%%%%%%%%%%%%%%%%%%%%%%%%%%%%%%%%%%%%%%%%%%%
\usepackage{newfloat}
\DeclareFloatingEnvironment[fileext=los,listname=List of Algorithms,name=Algorithm,placement=ht,within=chapter]{algo}
%\usepackage[font={sf,small},labelfont=bf]{caption}
%\captionsetup[figure]{position=above}
\setlength{\intextsep}{15pt plus 3pt minus 2pt}
%\setlength{\floatsep}{5pt plus 3pt minus 2pt}
%\setlength{\textfloatsep}{5pt plus 3pt minus 2pt}
\newsavebox{\coloredbgbox}%
\newenvironment{algobox}%
  {\begin{lrbox}{\coloredbgbox}\begin{minipage}[t]{\textwidth-2\fboxsep-2\fboxrule}\small}%
  {\end{minipage}\end{lrbox}%
   \centering\colorbox{blue!75!black!8}{\usebox{\coloredbgbox}}}%
%%%%%%%%%%%%%%%%%%%%%%%%%%%%%%%%%%%%%%%%%%%%%%%%%%%%%%%%%%%%%%%%%%%%%%%%%%%%%
% fancyhdr parameters
\pagestyle{fancy}
\renewcommand{\chaptermark}[1]{\markboth{\sc\thechapter. #1}{}}
\renewcommand{\sectionmark}[1]{\markright{\thesection\hspace{5pt}#1}{}}
\fancyhf{}
\fancyhead[LE,RO]{\normalfont\normalsize\bfseries\thepage}
\fancyhead[LO]{\rightmark}
\fancyhead[RE]{\leftmark}
\renewcommand{\headrulewidth}{0.5pt}
\addtolength{\headheight}{2.5pt}
\renewcommand{\footrulewidth}{0pt}
\fancypagestyle{plain}{\fancyhead{}\renewcommand{\headrulewidth}{0pt}}
%%%%%%%%%%%%%%%%%%%%%%%%%%%%%%%%%%%%%%%%%%%%%%%%%%%%%%%%%%%%%%%%%%%%%%%%%%%%%
\newcommand*\D{\mathop{}\!\mathrm{d}}
%\newcommand*\diff{\mathop{}\!\mathrm{d}}
\newcommand{\rem}{\textcolor{red}}
\newcommand{\Disp}{\displaystyle}
\newcommand{\expp}{\mathrm{e}}
\newcommand{\vx}{\mathbf{x}}
\newcommand{\vu}{\mathbf{u}}
\newcommand{\vy}{\mathbf{y}}
\newcommand{\vf}{\mathbf{f}}
\newcommand{\mA}{\mathbf{A}}
\newcommand{\fl}{\mathrm{fl}}
\newcommand{\Cont}{\mathcal{C}^0} % C^0 continuity
\newcommand{\ContC}{\mathcal{C}^1} % C^1 continuity
\newcommand{\ContCC}{\mathcal{C}^2} % C^2 continuity
\newcommand{\Contn}{\mathcal{C}^n} % C^n continuity
\newcommand{\base}{\mathsf{b}}
\DeclareMathOperator{\odivide}{\kern.5pt\ominus\kern-9.2pt\div\kern.5pt}
\DeclareMathOperator{\bmax}{\mathbf{max}}
\DeclareMathOperator{\bdiv}{\mathbf{div}}
\DeclareMathOperator*{\argmin}{arg\,min}
%%%%%%%%%%%%%%%%%%%%%%%%%%%%%%%%%%%%%%%%%%%%%%%%%%%%%%%%%%%%%%%%%%%%%%%%%%%%%
\newcommand{\bigO}{O} % big O
%\newcommand{\bigO}{\mathscr{O}} % big O
%\newcommand{\nsubset}{\not\subset}
%%%%%%%%%%%%%%%%%%%%%%%%%%%%%%%%%%%%%%%%%%%%%%%%%%%%%%%%%%%%%%%%%%%%%%%%%%%%%
\newcommand{\inteoo}[2]{\mathopen{(}#1\,;#2\mathclose{)}}
\newcommand{\inteff}[2]{\mathopen{[}#1\,;#2\mathclose{]}}
\newcommand{\inteof}[2]{\mathopen{(}#1\,;#2\mathclose{]}}
\newcommand{\intefo}[2]{\mathopen{[}#1\,;#2\mathclose{)}}
\newcommand{\intg}[2]{\mathopen{(}#1\,;#2\mathclose{)}}
%%%%%%%%%%%%%%%%%%%%%%%%%%%%%%%%%%%%%%%%%%%%%%%%%%%%%%%%%%%%%%%%%%%%%%%%%%%%%
\usepackage{listings}
\definecolor{mygreen}{RGB}{28,172,0} % color values Red, Green, Blue
\definecolor{mylilas}{RGB}{170,55,241}

\definecolor{mycolor1}{rgb}{0.00000,0.44700,0.74100}%
\definecolor{mycolor2}{rgb}{0.85000,0.32500,0.09800}%
\definecolor{mycolor4}{rgb}{0.49400,0.18400,0.55600}%
\definecolor{mycolor3}{rgb}{0.9290    ,0.6940    ,0.1250}
\DeclareRobustCommand{\colorlineA}{\tikz[baseline=-\the\dimexpr\fontdimen22\textfont2\relax,inner sep=0pt]\draw[mycolor1,solid,line width=1pt](0,0) -- (5mm,0);}
\DeclareRobustCommand{\colorlineB}{\tikz[baseline=-\the\dimexpr\fontdimen22\textfont2\relax,inner sep=0pt]\draw[mycolor2,solid,line width=1pt](0,0) -- (5mm,0);}
\DeclareRobustCommand{\colorlineC}{\tikz[baseline=-\the\dimexpr\fontdimen22\textfont2\relax,inner sep=0pt]\draw[mycolor3,solid,line width=1pt](0,0) -- (5mm,0);}
\DeclareRobustCommand{\colorADashed}{\tikz[baseline=-\the\dimexpr\fontdimen22\textfont2\relax,inner sep=0pt]\draw[mycolor1,dashed,line width=1pt](0,0) -- (5mm,0);}
\DeclareRobustCommand{\colorBDashed}{\tikz[baseline=-\the\dimexpr\fontdimen22\textfont2\relax,inner sep=0pt]\draw[mycolor2,dashed,line width=1pt](0,0) -- (5mm,0);}
\DeclareRobustCommand{\colorCDashed}{\tikz[baseline=-\the\dimexpr\fontdimen22\textfont2\relax,inner sep=0pt]\draw[mycolor3,dashed,line width=1pt](0,0) -- (5mm,0);}
\DeclareRobustCommand{\colorDDashed}{\tikz[baseline=-\the\dimexpr\fontdimen22\textfont2\relax,inner sep=0pt]\draw[mycolor4,dashed,line width=1pt](0,0) -- (5mm,0);}
\DeclareRobustCommand{\blackSolid}{\tikz[baseline=-\the\dimexpr\fontdimen22\textfont2\relax,inner sep=0pt]\draw[black,solid,line width=1pt](0,0) -- (5mm,0);}
\DeclareRobustCommand{\blackDashed}{\tikz[baseline=-\the\dimexpr\fontdimen22\textfont2\relax,inner sep=0pt]\draw[black,dashed,line width=1pt](0,0) -- (5mm,0);}
\DeclareRobustCommand{\blackDashdotted}{\tikz[baseline=-\the\dimexpr\fontdimen22\textfont2\relax,inner sep=0pt]\draw[black,dashdotted,line width=1pt](0,0) -- (5mm,0);}
\DeclareRobustCommand{\colorlineDashedBlack}{\tikz[baseline=-\the\dimexpr\fontdimen22\textfont2\relax,inner sep=0pt]\draw[black,dashed,line width=1pt](0,0) -- (5mm,0);}
\DeclareRobustCommand{\colorlineBDashed}{\tikz[baseline=-\the\dimexpr\fontdimen22\textfont2\relax,inner sep=0pt]\draw[mycolor2,dashed,line width=1pt](0,0) -- (5mm,0);}


\lstset{language=matlab,%
    %%%%%%%%%%%%%%%%%%%basicstyle=\color{red},
    breaklines=true,%
    morekeywords={matlab2tikz},
    keywordstyle=\color{blue},%
    morekeywords=[2]{1}, keywordstyle=[2]{\color{black}},
    identifierstyle=\color{black},%
    stringstyle=\color{mylilas},
    commentstyle=\color{mygreen},%
    showstringspaces=false,%without this there will be a symbol in the places where there is a space
    numbers=left,%
    numberstyle={\tiny \color{black}},% size of the numbers
    numbersep=9pt, % this defines how far the numbers are from the text
    %emph=[1]{for,end,break},emphstyle=[1]\color{red}, %some words to emphasise
    %%%%%%%%%%%%%%%emph=[2]{word1,word2}, emphstyle=[2]{style}, 
    flexiblecolumns=true,
	stringstyle=\ttfamily\footnotesize, % typewriter type for strings
	basicstyle=\ttfamily\footnotesize, % typewriter type for strings
	frame=single,
	backgroundcolor=\color{blue!75!black!8},
	rulecolor=\color{blue!50},
	framesep=0pt,
	rulesep=0pt,
	framerule=0pt,
	framexleftmargin=0pt,
}
